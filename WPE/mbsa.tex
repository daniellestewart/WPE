\subsection{Model Based Safety Analysis}
\label{sec:mbsa}

In model based safety analysis, the system and safety engineers share a common system model created using the model based development process. By extending the system model and relevant physical control systems, automated support can be provided for much of the safety analysis. Using a common model for both system and safety engineering and automating parts of safety analysis assists in the reduction of cost and improves the quality of the safety analysis. Safety engineers traditionally perform safety analysis based on information synthesized from a variety of sources including informal design models and requirement documents. These analyses are highly subjective and dependent on the skill of the analyst. The lack of precise models requires the analyst to devote a fair amount of time to information gathering of the architecture and behavior of the system. 

In model based system development, various development activities such as simulation, verification, testing, and code generation are based on a formal model of the system under development\cite{Joshi05:Dasc}. This is called the nominal model. Model based development was extended to include model based safety analysis\cite{Joshi05:Dasc,Joshi05:SafeComp,Joshi07:Hase,DBLP:conf/cav/BozzanoCPJKPRT15,CAV2015:BoCiGrMa,info17:HaLuHo}. This incorporates safety analysis into the model based development process in order to provide information on the safety of the formal model of the system under development. In this process, the nominal (non-failure) system behavior that is captured in the model based development process is augmented with the fault behavior of the system. Model based safety analysis then operates on a formal model that describes both nominal system behavior and the fault model which describes fault behavior. Over the last decade, model based safety analysis research has grown significantly. 





